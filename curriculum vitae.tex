\documentclass{resume} % Use the custom resume.cls style

\usepackage{hyperref}
\usepackage{xcolor}
\usepackage{amsmath,amsthm,amssymb,amsfonts,setspace}
\usepackage[shortlabels]{enumitem}

\newcommand{\al}{\alpha}
\newcommand{\be}{\beta} 
\newcommand{\del}{\delta} 
\newcommand{\Del}{\Delta}
\newcommand{\lam}{\lambda}  
\newcommand{\Lam}{\Lambda} 
\newcommand{\ep}{\epsilon}
\newcommand{\sig}{\sigma} 
\newcommand{\om}{\omega}
\newcommand{\Om}{\Omega}
\newcommand{\C}{\mathbb{C}}
\newcommand{\N}{\mathbb{N}}
\newcommand{\E}{\mathbb{E}}
\newcommand{\Z}{\mathbb{Z}}
\newcommand{\R}{\mathbb{R}}
\newcommand{\Q}{\mathbb{Q}}
\renewcommand{\P}{\mathbb{P}}
\newcommand{\MA}{\mathcal{A}}
\newcommand{\MB}{\mathcal{B}}
\newcommand{\MF}{\mathcal{F}}
\newcommand{\MG}{\mathcal{G}}
\newcommand{\MJ}{\mathcal{J}}
\newcommand{\ML}{\mathcal{L}}
\newcommand{\MN}{\mathcal{N}}
\newcommand{\MS}{\mathcal{S}}
\newcommand{\MP}{\mathcal{P}}
\newcommand{\ME}{\mathcal{E}}
\newcommand{\MT}{\mathcal{T}}
\newcommand{\MM}{\mathcal{M}}

\usepackage[left=0.75in,top=0.6in,right=0.75in,bottom=0.6in]{geometry} % Document margins
\newcommand{\tab}[1]{\hspace{.2667\textwidth}\rlap{#1}}
\newcommand{\itab}[1]{\hspace{0em}\rlap{#1}}
\name{Carson James} % Your name
\address{(+1)~405~315~5881 \\ carson.a.james@gmail.com} % Your phone number and email
\address{201 Darwin Rd \\ Edmond, Oklahoma 73034} % Your address
%\address{123 Pleasant Lane \\ City, State 12345} % Your secondary addess (optional)


\begin{document}

%----------------------------------------------------------------------------------------
%	Education
%----------------------------------------------------------------------------------------
\begin{rSection}{Education}
MSc Mathematics \hfill {\em August 2016 - May 2018} \\
{\bf Oklahoma State University}  \hfill {GPA: 4.0} 

BA Mathematics \hfill {\em August 2013 - May 2016} \\
{\bf Oklahoma State University} \hfill {GPA: 3.929} \\
\end{rSection}

%----------------------------------------------------------------------------------------
%	Research Interests
%----------------------------------------------------------------------------------------
\begin{rSection}{Research Interests and Current Projects}

{\bf Time Series of Economic and Financial Data:}
\begin{itemize}
\item I made a stock screener in python that pulls price data from the iex api and creates a graph with nodes and edges between nodes corresponding to tickers and conintegrated tickers respectively. I'm currently in the process of moving this to R and adding some GARCH modeling functionality. (\href{https://github.com/carsonaj/Finance/tree/master/market}{\color{blue} screener})

\item I recently began making some notes on modeling financial data and risk. I am working through various sources like the 2000 and 2002 papers of Rockafellar and Uryasev detailing conditional value at risk and its equivalent forms, the 2008 paper by Chen detailing nonparametric estimation of conditional value at risk and the 2003 paper by Angelidis, Benos and Degiannakis detailing the estimation of value at risk using GARCH models.(\href{https://github.com/carsonaj/Math/blob/master/Portfolio%20Theory/Portfolio%20Theory%20Notes.pdf}{\color{blue} notes})
\end{itemize}

{\bf Stochastic Processes:}
\begin{itemize}
\item I am currently in picking up stochastic integration with semimartingales and compiling some notes with exercises that might be useful for other students who are new to the area like me and would benefit from a pointed introduction. I was previously consulting various sources but recently I discovered the book {\em Limit Theorems for Stochastic Processes} by Jacod, Shiryaev and am now rewriting my notes. (\href{https://github.com/carsonaj/Math/blob/master/Stochastic%20Analysis/Stochastic%20Processes%20-%20James.pdf}{{\color{blue} notes}}) 
\end{itemize}

{\bf Arithmetic Dynamics:}  
\begin{itemize}
\item A requirement of my masters degree consisted in creating some introductory notes to some open problems in the area of arithmetic dynamics. The focus is centered on introducing the notion of height of algebraic numbers,  potential theory and the interplay between the two. In particular, given some polynomial $\phi \in \Z[z]$ with $deg(\phi) \geq 2$, we can consider the Julia set $\MJ_{\phi}$ of $\phi$, the canonical height $\hat{h}_\phi$ associated with $\phi$ and the equilibrium measure $\mu$ of $\MJ_{\phi}$, that is, the measure that minimizes the energy functional $\int_{\MJ_{\phi}^2}-log|x-y|d\nu^2$ over all Borel probability measures $\nu$ with support in $\MJ_{\phi}$. Then any sequence $(z_n)_{n \in \N} \subset \overline{\Q}$ with $deg(z_n) \rightarrow \infty$ and $\hat{h}_{\phi}(z_n) \rightarrow 0$ as $n \rightarrow \infty$ has the conjugates of $z_n$ equidistributing around $\MJ_\phi$. There are open problems regarding the existence of a lower bound for $\hat{h}_{\phi}(z)$ for $z$ not preperiodic and in some sense bounded, but there is no answer for even simple cases like $\phi(z) = z^2+c$ with $c \in \Z$. I periodically update the notes. (\href{https://github.com/carsonaj/Math/blob/master/Arithmetic%20Dynamics/Arithmetic%20Dynamics%20Notes.pdf}{{\color{blue} creative component}})
\end{itemize}

\newpage
\end{rSection}

%----------------------------------------------------------------------------------------
%	Work Experience
%----------------------------------------------------------------------------------------
\begin{rSection}{Work Experience}
{\bf Math Teacher} at Pensacola High School \hfill {\em August 2019 - Present}\\
{\bf Courses Taught:} 
\begin{itemize}
\item Honors Algebra II
\item Honors Precalculus 
\item IB Statistics
\end{itemize} 

{\bf Graduate Teaching Assistant} at Oklahoma State University \hfill {\em August 2016 - May 2018} \\
{\bf Courses Taught:} 
\begin{itemize}
\item Trigonometry (instructor of record)
\item Business Calculus (recitation)
\end{itemize} 

\end{rSection}

%----------------------------------------------------------------------------------------
%	Volunteer Experience
%----------------------------------------------------------------------------------------
\begin{rSection}{Volunteer Experience}
{\bf Volunteer} with Love Without Boundaries Cambodia \hfill {\em May 2018 - September 2018}\\
{\bf Responsibilities:} 
\begin{itemize}
\item Taught English to grades 11 and 12 in Tuol Prasat High School, 
\item Assisted the LWB staff in writing donor reports.
\end{itemize}
\end{rSection}

%----------------------------------------------------------------------------------------
%	Skills 
%----------------------------------------------------------------------------------------
\begin{rSection}{Skills}

{\bf Computer Languages}
\begin{itemize}
\item Python (intermediate)
\item C (intermediate)
\item R (basic)
\item SQL (basic)
\end{itemize}

{\bf Languages}
\begin{itemize}
\item English (native)
\item Spanish (fluent)
\end{itemize}

\end{rSection}

%----------------------------------------------------------------------------------------
%	Awards and Honors
%----------------------------------------------------------------------------------------
\begin{rSection}{Awards and Honors}
Hazel Bucy Endowment Fund (2017)

Member of Phi Beta Kappa Honor Society (2016)

Litchenburg Family Scholarship for Mathematics (2014)

Department of Mathematics General  Scholarship (2014)
\end{rSection}

%----------------------------------------------------------------------------------------
%	Awards and Honors
%----------------------------------------------------------------------------------------
\begin{rSection}{References}
Paul Fili, Department of Mathematics, Oklahoma State University, paul.fili@okstate.edu

Alan Noell, Department of Mathematics, Oklahoma State University, noell@math.okstate.edu

Igor Pritsker, Department of Mathematics, Oklahoma State University, igor@math.okstate.edu
\end{rSection}



\end{document}